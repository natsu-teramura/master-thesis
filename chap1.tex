\chapter{序論}
\thispagestyle{empty}
\label{chap:1}
素粒子の標準模型~(Standard~Model:~SM)は現代において最も成功した素粒子物理学の理論である~\cite{URL:10}。
強い相互作用における量子色力学~\cite{AR:04}、弱い相互作用と電磁相互作用について記述したワインバーグ=サラム理論~\cite{AR:05,AR:05a}、ヒッグス機構による真空の対称性の破れとフェルミオンの質量獲得~\cite{AR:06}、CP~対称性の破れについて記した小林・益川理論~\cite{AR:07}などの理論はすべて標準模型に内包されたものである。
標準模型を構成する素粒子の一覧を\tbref{tb:SM1}と\tbref{tb:SM2}に示す。
これらの素粒子は、物質を構成する粒子~(フェルミオン)~と力を媒介する粒子~(ボソン)~の~2~種類に区分することができる。フェルミオンは~6~種のクォークと~6~種のレプトンから構成され、強い相互作用をするものをクォーク、しないものをレプトンとして大別している。ボソンには相互作用をつかさどる~4~種のゲージボソンと質量の起源となるヒッグス粒子が存在する。

素粒子に関するあらゆることを記述することができる標準模型であるが、多くの未解決問題を含んでいることも事実である。
現代の素粒子物理学において標準模型をより深く理解し、標準模型を超える理論~(Beyond~the~Standard~Model:~BSM)~を探究することは、物理学の発展の上で最も重要なことの一つである。

スイスのジュネーブに拠点を置く欧州原子核研究機構~(CERN)~\cite{URL:11}では、素粒子物理のさらなる発展を実現するために様々な研究活動が行われている。Large~Hadron~Collider~(LHC)~\cite{URL:12}は、CERN~によって建設された世界最高エネルギーの陽子陽子衝突型加速器で、A~Toroidal~LHC~ApparatuS~(ATLAS)~実験~\cite{URL:13}は~LHC~の衝突点の一つで行われている世界最大規模の素粒子実験である。
ATLAS~実験や同じく~LHC~で行われている~Compact~Muon~Solenoid~(CMS)~実験~\cite{URL:14}は、2012~年にヒッグス粒子の発見に成功し現代の素粒子物理学の発展に大きく貢献した~\cite{TR:03,TR:03a}。
LHC~では、2022~年から始まる第三期運転~(Run~3)~に向けたアップグレードが行われている。Run~3~では超対称性粒子や暗黒物質の探索など、宇宙誕生の謎を解明する新発見を目指している。

\begin{table}[htbp]
	\centering
	\begin{tabular}{c|ccc|c|c} \hline
	& \multicolumn{3}{c|}{記 号} & \multirow{2}{*}{スピン} & \multirow{2}{*}{電  荷} \\
	& 第1世代 & 第2世代 & 第3世代 &&  \\ \hline\hline
	\multirow{2}{*}{クォーク} & \it{u} & \it{c} & \it{t} & \multirow{2}{*}{1/2} & +2/3 \\
	& \it{d} & \it{s} & \it{b} &  & -1/3 \\ \hline
	\multirow{2}{*}{レプトン} & $\nu_{e}$ & $\nu_{\mu}$ & $\nu_{\tau}$ & \multirow{2}{*}{1/2} & 0 \\
	& $\it{e}^-$ & $\mu^-$ & $\tau^-$ &  & -1 \\ \hline
	\end{tabular}
	\caption[標準模型を構成するフェルミオンの一覧]{標準模型を構成するフェルミオンの一覧~\cite{URL:10}。}
	\label{tb:SM1}
\end{table}

\begin{table}[htbp]
	\centering
	\begin{tabular}{c|c|c|c|c}\hline
	&記 号& スピン & 電 荷 &\\ \hline\hline
	\multirow{4}{*}{ゲージボソン} & $\gamma$ & 1 & 0 & 電磁相互作用 \\
	 & $g$ & 1 & 0 & 強い相互作用 \\
	 & $W^{\pm}$ & 1 & $\pm1$ & \multirow{2}{*}{弱い相互作用} \\
	 & $Z$ & 1 & 0 &  \\ \hline
	スカラーボソン & $H$ & 0 & 0 & 質量の起源 \\ \hline
	\end{tabular}
	\caption[標準模型を構成するボソンの一覧]{標準模型を構成するボソンの一覧~\cite{URL:10}。}
	\label{tb:SM2}
\end{table}

2018~年まで行われていた~ATLAS~実験第二期運転~(Run~2)~では新物理の発見を目指し様々な探索が行われていたが、いくつかの課題も残っている。
陽子衝突により生成された大量の粒子から興味のある事象を選別~(トリガー)~するには、異なるタイミングで複数の検出器に飛来してくる同一事象の粒子を一括して読み取らなければならない。そのためには各検出器の信号読み出しのタイミングを合致させる必要がある。Thin~Gap~Chamber~(TGC)~\cite{TR:04}は、トリガー用検出器の一つでありタイミングの調整が非常に重要となるが、Run~2~ではタイミングの見積もりが不十分であったことによるヒット効率の位置依存性が確認された。

本論文は~TGC~検出器に着目し詳細なタイミング検証を行うことで性能改善を行い、ATLAS~実験における~SM~の精密探索や新物理探索に寄与することを目的としている。
本論文は以下のように~TGC~検出器における性能検証やトリガー性能の評価およびそれに伴った新物理に対する評価方法の構築などについてまとめている。

\chapref{chap:1}には研究背景および論文構成について記している。\chapref{chap:2}では~LHC-ATLAS~実験の詳細および各検出器の役割について記し、ATLAS~実験の概要を説明する。\chapref{chap:3}では本論文において着目している~TGC~検出器の詳細な役割および構成についての説明を行う。\chapref{chap:4}では新物理探索の一つとして行われている重い長寿命荷電粒子の探索について述べ、本研究の目的を記した。\chapref{chap:5}では~TGC~検出器のタイミングに関する調査および詳細なタイミング較正の結果について示す。\chapref{chap:6}ではタイミング較正に伴った新たなトリガー効率の見積もり手法の構築や性能改善の結果についてまとめた。最後に、\chapref{chap:7}では結論および今後の展望について述べる。