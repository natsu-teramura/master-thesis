%\thispagestyle{empty}
\begin{center}\vspace*{-2cm}
    \textbf{令和3年度 修士論文発表会概要}\\
    \large{LHC-ATLAS~実験におけるトリガー用前後方ミューオン検出器の}\\
    \large{詳細なタイミング較正による性能改善}
\end{center}
%\vspace{-3cm}
\begin{flushright}
粒子物理研究室 207s114s 寺村~七都\\
指導教員 前田~順平
\end{flushright}
%\vspace{-3cm}
\thispagestyle{empty}
%\vspace{10pt}
欧州原子核研究機構~(CERN)~では、素粒子物理学の発展を目指し様々な研究が行われている。CERN~が建設したLarge~Hadron~Collider~(LHC)~と呼ばれる陽子陽子衝突型円形加速器は、世界最高のエネルギーを誇っており~2012~年には標準模型で唯一未発見であったヒッグス粒子の発見に至った。ATLAS~実験は~LHC~の衝突点の一つで行われている実験であり、国内外約~5000~人の研究者が標準模型の精密測定や新物理の探索に力を注いでいる。2022~年に~ATLAS~実験第三期運転が開始される予定で、運転に向けた改良・調整も佳境を迎えている。

ATLAS~実験は大型の汎用測定器を用いており、最外部に位置するミューオン検出器は、新物理探索のプローブとなるミューオンをとらえる上で非常に重要な役割を担っている。中でも~Thin~Gap~Chamber~(TGC)~検出器は、ATLAS~検出器の前後方に位置し、高速に事象選別~(トリガー)~を行う上で欠かすことのできない装置となっている。
LHC~では40~MHzという非常に高い頻度での陽子衝突が行われている。トリガーを行うには複数の検出器に異なるタイミングで飛来してくる同一事象の粒子を一括して読み取ることが必要となる。
各検出器における同一事象の情報を一致させるには、タイミングの調整が大切であるが、粒子信号を読み取るまでには様々な過程があり、検出器の位置やケーブルの長さ等を考慮しなくてはならない。

本研究では詳細なタイミング検証により、TGC~検出器の性能評価を行った。ATLAS~実験第二期運転で収集されたデータとモンテカルロシミュレーションを比較・検証することで、複数の問題点を明らかにすることに成功し第三期運転に向けた改良を施した。

そして、TGC~検出器のタイミング較正は新物理探索においても重要な意味を持つ。
新物理探索の一つである超対称性粒子は、通常の粒子と比べ質量が非常に重く粒子速度が遅いことが予測されている。
またいくつかのモデルでは崩壊先が抑制され、ミューオン検出器に到達するほどの長い寿命であることも示唆されている。
ATLAS~では速度の遅い粒子をとらえるために新たな探索用トリガーの導入を行った。速度の遅い粒子は、光速の粒子と比べ検出器間の飛来時間差が大きくなるため、事象を正しく同定するにはより詳細なタイミングの見積もりが必要となる。
本研究では新粒子のシミュレーションサンプルを用い、タイミング調整に伴うトリガー性能の評価を行った。また実験データにおいては、新物理が未観測のため直接的なトリガー効率の算出が行えない。そのためTGC~検出器でのタイミング判定をもとにした新たなトリガー効率の概算手法を提案した。

本発表では、TGC~検出器の詳細なタイミング検証による第三期運転に向けた性能改善および重い長寿命荷電粒子の探索に特化したトリガーの性能と新たに開発した評価方法について講演する。