\chapter{結論と展望}
\thispagestyle{empty}
\label{chap:7}
LHC~では、2022~年から第三期運転が開始される予定であり、ATLAS~実験においても標準模型の精密測定や新物理の発見を目指し、様々なアップグレードが行われている。
ATLAS~検出器の一つである~TGC~検出器は、ミューオンなどの荷電粒子をとらえ、トリガー判定を行う重要な役割を担っている。
本論文では、ATLAS~実験~Run~3~に向けた~TGC~検出器の性能改善のための詳細なタイミング較正の結果を示した。また、新物理探索のための新たなトリガー効率の見積もり手法の構築および検証を行った。

TGC~検出器でのヒットを受けて一時的に保持された陽子バンチごとの粒子情報は、L1~トリガーからの出力信号により、順番に選別される。
検出器でのヒット信号を正しくトリガーするためには信号に対するバンチ判定を適切なタイミングで行い、各検出器でのタイミングの一致をとることが非常に重要である。本研究では~Run~2~の実験データおよびモンテカルロシミュレーションを用いて~TGC~検出器の詳細なタイミング検証を行った。バンチ判定の情報からタイミングの評価を行うためのパラメータを定義し、実験データとシミュレーションにおいてタイミング判定に差異がみられることを確認した。そこで、Run~3~に向けて~TGC~検出器の性能向上を目指し、バンチ判定判定における詳細な調査およびシミュレーションの改良を行った。シミュレーションにおける改良の一点目は、信号検出のためのセンサーから信号を読み出す~ASD~までの信号伝搬計算の実装である。各チェンバーに設置されている~ASD~の配置と対応するチャンネルをデータベース化し、伝搬時間の計算システムを構築した。二点目は、ツイストケーブルの半径に依存したタイミング遅延の実装である。セクターごとのツイストケーブルの長さとケーブル半径から遅延時間差の計算を行った。三点目は、信号伝搬に伴ったシグナルの減衰による影響の実装である。信号減衰の実測値をもとに、ケーブル長と遅延時間の関係式を導いた。以上の改良により、TGC~検出器の詳細なタイミング較正に成功した。また較正に伴い、タイミングがヒット効率の位置依存性に影響することを示唆した。チェンバーの一部においては構造上の関係により~ASD~が端によって設置されていた。ASD~の位置に依存したタイミングの遅延が生じていることを示し、Run~2~の実験データにおけるヒット効率をタイミング較正によって再現することに成功した。そして、ヒット効率の位置依存性を解消するための遅延パラメータの再調整を行った。

さらに本研究では、タイミング較正に伴ったトリガー性能の評価を行った。まずは光速のミューオンに対するトリガー性能の評価を行い、Run~2~の実験データと良く一致していることを示した。また~LHC~では新物理探索の一つとして重い長寿命荷電粒子の探索が進められている。重い長寿命荷電粒子は、超対称性理論のいくつかのモデルによって存在が示唆されている。本研究では~GMSB~モデルにより存在が予想されているスタウ粒子のシミュレーションサンプルを用いたトリガー性能評価を行った。評価に用いたトリガーは、標準的なシングルミューオントリガーおよび遅い荷電粒子探索用トリガーである。遅い荷電粒子探索用トリガーは~Run~2~後半に新たに導入された次のバンチと判定された速度の遅い粒子を対象としたトリガーである。タイミング較正前後におけるトリガー効率を比較し、タイミング較正の影響により、獲得できる粒子速度の領域に違いがみられることを示した。シミュレーションによって示したトリガー効率を評価するためには、実験データにおけるトリガー効率を算出する必要がある。しかし、実験においてスタウ粒子の事象は未観測であり、直接的なトリガー効率の算出は行うことができない。そこで、バンチ判定の情報からヒット信号の時間に依存した確率分布関数を定義することで、粒子速度に依存したトリガー効率を見積もる新たな解析手法を確立した。シミュレーションを用いることで新たな解析手法の検証を行い、構築したトリガー効率の見積もり手法を用いて~Run~2~のデータおよびタイミング較正前後のシミュレーションにおけるトリガー性能の評価を行った。

Run~3~では新しい検出器の導入などの様々なアップグレードが予定されている。TGC~検出器のタイミング調整に関しても改めて較正を行う必要がある。
本研究で行った~TGC~におけるタイミング評価方法の構築および詳細なシミュレーションの改良によって、Run~3~ではより詳細なタイミング設定が行えることが期待できる。また、重い長寿命荷電粒子の探索においても新たに構築したトリガー効率の評価方法を利用することで、物理解析に寄与できることを期待したい。