\chapter{結論と展望}
\thispagestyle{empty}
\label{chap:7}
LHC~では、2022~年から第三期運転が開始される予定であり、ATLAS~実験においても標準模型の精密測定や新物理の発見を目指し、様々なアップグレードが行われている。
ATLAS~検出器の一つである~TGC~検出器はミューオンなどの荷電粒子をとらえ、トリガー判定を行う上で重要な役割を担っている。
本論文では、ATLAS~実験~Run~3~に向けた~TGC~検出器の性能改善のための詳細なタイミング較正の結果を示した。またバンチ判定を利用して、新物理探索のための新しいトリガー効率見積もり手法の構築および検証を行った。

TGC~検出器でのヒットを受けて一時的に保持された陽子バンチごとの事象の情報は、L1~トリガーからの出力信号により、順番に選別される。
事象を正しくトリガーするためには信号に対するバンチ判定を適切なタイミングで行い、各検出器でのタイミングの一致をとることが非常に重要である。本研究では~Run~2~の実験データおよびモンテカルロシミュレーションにおいて~TGC~検出器の詳細なタイミング検証を行った。バンチ判定の情報からタイミングの評価を行うための指標を定義し、実験データとシミュレーションにおいてタイミング判定に差異がみられることを確認した。そこで、Run~3~に向けた~TGC~検出器の性能向上を目指し、バンチ判定における詳細な調査およびシミュレーションの改良を行った。シミュレーションにおける改良の一点目は、信号検出のためのセンサーから信号を読み出す~ASD~までの信号伝搬計算の実装である。二点目はツイストケーブルの半径に依存したタイミング遅延の追加で、三点目は信号伝搬に伴った信号の減衰による効果の実装である。以上の改良により、TGC~検出器の詳細なタイミング較正に成功した。また較正に伴いタイミングがヒット効率の位置依存性に影響することを示唆し、実験データのヒット効率を再現することに成功した。そして~Run~3~に向けてヒット効率の位置依存性を解消するための遅延パラメータの再調整を行った。
さらに本研究ではタイミング較正に伴ったトリガー性能の評価を行った。光速のミューオンに対するトリガー性能の評価を行い、シミュレーションにおけるトリガー効率が~Run~2~の実験データと良く一致していることを示した。

また新物理探索に向けたトリガー性能についても評価を行った。
LHC~では新物理探索の一つとして重い長寿命荷電粒子の探索が進められている。重い長寿命荷電粒子は、超対称性理論のいくつかのモデルによって存在が示唆されており、本研究では~GMSB~モデルにより存在が予測されている長寿命スタウ粒子のシミュレーションサンプルを用いたトリガー性能評価を行った。Run~2~の後半に新たに導入された遅い荷電粒子探索用トリガーは次バンチと判定された速度の遅い荷電粒子を対象としたトリガーである。タイミング較正前後におけるスタウ粒子のトリガー効率を比較し、タイミング較正の影響により、獲得できる粒子速度の領域に違いがみられることを示した。また実験においてスタウ粒子は未観測であり直接的なトリガー効率の算出は行うことができないため、バンチ判定の情報からヒット信号の時間に依存した確率分布関数を定義することで、粒子速度に依存したトリガー効率を見積もる新たな解析手法を確立した。シミュレーションを用いることで新たな解析手法の検証を行い、構築したトリガー効率の見積もり手法を利用して~Run~2~のデータおよびタイミング較正前後のシミュレーションにおけるトリガー性能の評価を行った。

また本研究においてはいくつかの課題も残っている。TGC~検出器のタイミング較正に関しては、チャンネル単位でのタイミングの細かな差異等に改善の余地がある。またトリガー効率の見積もり手法においては正当性の評価のために、試験を行う必要がある。Run~3~に向けた今後の課題としてここに言及しておく。

Run~3~では新しい検出器の導入などの様々なアップグレードが予定されている。TGC~検出器のタイミング調整に関しても改めて較正を行う必要がある。
本研究で行った~TGC~におけるタイミング評価方法の構築および詳細なシミュレーションの改良によって、Run~3~ではより詳細なタイミング設定が行えることが期待できる。重い長寿命荷電粒子の探索においても新たに構築したトリガー効率の評価手法により、解析感度の向上に新たな道を切り拓いた。